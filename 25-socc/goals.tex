\section{Goals \& Assumptions}
\label{sec:goals}

\parhead{Threat model.}
%
Our \SystemName system has three parties: the \emph{cloud provider}, the
\emph{function providers}, and \emph{external attackers}.
%
A function chain may contain functions from multiple providers.
%
For any given function provider, the potential adversaries
are the other function providers in the chain, the cloud provider, and
external attackers.
%
The goal of an adversary is to learn the inputs or outputs of a (peer)
function, modify these inputs or outputs, or modify the sequence of functions
in the chain (e.g., to insert a function that sends data to an
adversary-controlled log).
%
A function provider can submit any function to the chain, including a
malicious function that tries to leak data or subvert the cloud
provider.
%
We assume that functions may contain bugs that unintentionally leak
data or expose the function to exploitation.
%
We assume an adversary cannot breach the security of confidential VMs; we trust
the system software in the VM, and consider side-channel attacks out-of-scope.
%
%Since the cloud provider can trivially deny service, we do not guarantee
%availability.


\parhead{Goals.}
%
% Security goals
% - confidentiality
% - integrity
% - key recovery (rotation)
%
Our primary security goals when designing \SystemName are:
%
\begin{itemize}
    \item[\textbf{S1}] \emph{Confidentiality}: 
        An attacker that breaches the FaaS platform or exploits a function must
        not be able to undermine the confidentiality of a peer function or its
        I/O\@.
        %
    \item[\textbf{S2}] \emph{Integrity}:
        An attacker that breaches the FaaS platform or exploits a function
        must not be able to alter the control flow of a function chain.
\end{itemize}
%
Additionally, we have the following functional goals:
%
\begin{itemize}
    \item[\textbf{F1}] \emph{Application transparency}: \SystemName must
        support unmodified functions.
        %
        In particular, a function developer must not have to modify their
        function to support \SystemName's security features.
        %
    \item[\textbf{F2}] \emph{Compatibility with existing FaaS platforms}:
        \SystemName must extend current FaaS platforms (our future work
        targets Knative).
        %
        This implies supporting the autoscaling policies of such platforms
        (including scaling down to zero replicas).
        %
    \item[\textbf{F3}] \emph{Preservation of Untrusted Control Plane}:
        \SystemName must not add trusted components to the FaaS control plane.
        %
    \item[\textbf{F4}] \emph{Low performance overheads}:
        \SystemName must impose only modest performance overhead on the
        FaaS applications, both in terms of client latency and
        resource usage.
\end{itemize}
