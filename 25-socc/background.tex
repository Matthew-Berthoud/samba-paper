\section{Background \& Motivation}
\label{sec:background}

\parhead{Confidential computing.}
%
\emph{Confidential computing} protects data in use by processing it within a
hardware TEE, isolating it from unauthorized access or modification.
%
This proposal focuses on the latest generation of TEEs---specifically AMD
SEV-SNP~\cite{amd-sev_snp}---which encrypts the memory of an entire guest VM to
create \emph{confidential VMs}.
%
% (the Versioned Chip Endorsement Key, or VCEK) 
A critical feature of AMD SEV-SNP (and all TEEs) is \emph{remote attestation}:
the trusted hardware measures (computes a hash of) the VM workload, and a
unique hardware key signs this initial state, producing an \emph{attestation
report}.
%
Thus, AMD serves as the root of trust, eliminating the need to trust the server
owner, while customers can retrieve the attestation report to verify that the
correct software is running within the confidential VM\@.
%
To allow an attestation to bind a runtime value, the confidential VM can
include a small amount of \emph{user data} in the signed report.
%
%
%AMD also provides certificate authority servers that can issue a VCEK
%certificate by signing the corresponding hardware key.


\parhead{Proxy re-encryption.}
%
% https://www.cs.jhu.edu/~susan/600.641/scribes/lecture17.pdf
%
\emph{Proxy re-encryption} is a cryptographic scheme that allows an untrusted
proxy to convert a ciphertext encrypted under Alice's public key into a
ciphertext that Bob can decrypt with his secret key, without learning the
underlying plaintext.
%
At a high-level, Alice and Bob construct a public \emph{re-encryption key}
$RK_{\text{Alice}\rightarrow\text{Bob}}$ that
the proxy uses to re-encrypt the ciphertext from Alice (the \emph{delegator}) to Bob
(the \emph{delegatee}).
%
In 1998, Blaze, Bleumer, \& Strauss~\cite{98-eurocrypt-proxy_cryptography}
designed the first proxy re-encryption construction based on the ElGamal
encryption system~\cite{85-toit-elgamal}.
%
Shortly thereafter, Dodis and Ivan~\cite{03-ndss-proxy_cryptography_revisited}
developed a unidirectional variant, and later Ateniese et
al.~\cite{05-ndss-improved_proxy_reencryption} applied bilinear
maps~\cite{01-crypto-ibe_weil_pairing} (and specifically BLS
signatures~\cite{03-eurocrypt-aggregate_signatures_bilinear_maps}) to develop
schemes that did not require interaction between the delegator and delegatee.
%
Since then, numerous works have explored features like multiple
re-encryptions~\cite{17-tops-fast_proxy_re_encryption} and
revocation~\cite{12-crypto-dynamic_credentials_and_delegation_for_abe}, and
security properties like chosen-ciphertext
resistance~\cite{07-ccs-cca_proxy_re_encryption} and
unlinkability of ciphertexts~\cite{19-acisp-pcs_proxy_reencryption}.


\subsection{Threat Model}
\label{sec:threat-model}

Our system has three parties: the \emph{cloud provider}, the \emph{function
providers}, and \emph{external attackers}.
%
A function chain may contain functions from multiple providers.
%
For any given function provider, the potential adversaries
are the other function providers in the chain, the cloud provider, and
external attackers.
%
The goal of an adversary is to learn the inputs or outputs of a (peer)
function, modify these inputs or outputs, or modify the sequence of functions
in the chain (e.g., to insert a function that sends data to an
adversary-controlled log).
%
A function provider can submit any function to the chain, including a
malicious function that tries to leak data or subvert the cloud
provider.
%
We assume that functions may contain bugs that unintentionally leak
data or expose the function to exploitation.
%
We assume an adversary cannot breach the security of confidential VMs; we trust
the system software in the VM, and consider side-channel attacks out-of-scope.
%
%Since the cloud provider can trivially deny service, we do not guarantee
%availability.


\subsection{Goals}
\label{sec:goals}

\color{red}
Diam euismod dictum placerat vehicula mollis. Malesuada proin ultricies dapibus
sem tincidunt suscipit nullam. Risus felis fusce quam hendrerit purus ultrices
fusce. Finibus etiam torquent nunc egestas fringilla habitasse phasellus mattis
pretium. Cursus finibus quisque imperdiet nunc gravida gravida elit. Luctus
aliquet erat tristique, felis mollis nullam. Dapibus at cras dolor consequat
egestas. Et nostra facilisis odio commodo hac et felis. Tempor ullamcorper
mollis pellentesque nulla hendrerit cursus magna praesent nulla. Porta cubilia
orci iaculis purus enim parturient.
\color{black}
