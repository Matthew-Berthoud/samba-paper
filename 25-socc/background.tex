\section{Background}
\label{sec:background}

\parhead{Function as a Service}
%
Function as a Service (FaaS) is a cloud computing model that enables developers
to deploy and execute individual functions or pieces of code without managing
the underlying infrastructure.
%
It is sometimes referred to as serverless computing, since these functions are
event-driven and run in stateless containers, automatically scaling up or down
based on demand, according to the policy of the orchestrator.
%
FaaS eliminates the need for developers to
provision and manage servers, allowing them to focus on writing code while the
cloud provider handles resource allocation, availability, and execution.
%
Popular FaaS platforms include AWS Lambda, Google Cloud Functions, and Azure
Functions.  Open and partially open source FaaS platforms include OpenFaaS,
OpenWhisk, and Open Function, as well as Knative,\footnote{\url{https://knative.dev}}
the open source serverless library to be used in \SystemName.


%\parhead{Confidential computing.}
%%
%\emph{Confidential computing} protects data in use by processing it within a
%hardware TEE, isolating it from unauthorized access or modification.
%%
%This proposal focuses on the latest generation of TEEs---specifically AMD
%SEV-SNP~\cite{amd-sev_snp}---which encrypts the memory of an entire guest VM to
%create \emph{confidential VMs}.
%%
%A critical feature of AMD SEV-SNP (and all TEEs) is \emph{remote attestation}:
%the trusted hardware measures (computes a hash of) the VM workload, and a
%unique hardware key signs this initial state, producing an \emph{attestation
%report}.
%%
%Thus, AMD serves as the root of trust, eliminating the need to trust the server
%owner, while customers can retrieve the attestation report to verify that the
%correct software is running within the confidential VM\@.
%%
%To allow an attestation to bind a runtime value, the confidential VM can
%include a small amount of \emph{user data} in the signed report.
%%
%%% %AMD also provides certificate authority servers that can issue a VCEK
%%% %certificate by signing the corresponding hardware key.


\parhead{Confidential computing.}
%
\emph{Confidential Computing} is executing a workload on an untrusted
third-party machine (the cloud) in a way that ``shields'' the workload from the
third-party: the third-party cannot alter, observe, or interfere with the
computation or its data (though, in practice, side channels~\cite{
    18-sec-foreshadow,
    19-sec-unprotected_io_in_sev,
    20-oakland-lvi,
    21-ccs-one_glitch,
    21-sec-cipherleaks,
    22-dissertation-amd_sev_flaws,
    22-sec-aepic_leak}
weaken these guarantees).
%
Confidential computing relies on hardware \emph{trusted execution environments} (TEEs) to
ensure memory isolation from the rest of the system: an application's memory is
encrypted and integrity-protected when in DRAM, and decrypted only when the
memory enters the trusted CPU package.
%
When a workload is running within a TEE, we say that the workload is running
within an \emph{enclave}; as shorthand, we often simply call the workload
an enclave.


%\parhead{Process- and VM-based enclaves.}
%
Intel SGX~\cite{intel_sgx_programming_reference,13-hasp-innovative_instructions}
was the first general approach to confidential computing.
%
By representing enclaves at the granularity of a part of a process, Intel SGX
promotes a small trusted computing base, but at the expense of
incompatibility with legacy software, though several middleware solutions~\cite{ 
14-osdi-haven, 16-osdi-scone, 17-atc-graphene, 20-sec-conclaves} attempt to
ease this burden.
%
To allow for unmodified applications to run within an enclave, the current
generation of TEEs, such as AMD
SEV~\cite{amd-memory_encryption,amd-sev_es,amd-sev_snp}, Intel
TDX~\cite{intel_tdx}, Arm CCA~\cite{introducing_arm_cca}, and IBM's
PEF~\cite{21-eurosys-openpower}, implements
enclaves at the granularity of a virtual machine (called \emph{confidential
VMs}).


%\parhead{Attestation.}
%
Two core properties of enclaves are
attestation~\cite{13-hasp-innovative_attestation} and sealing.
%
In \emph{attestation}, a hardware root-of-trust signs a measurement (digest)
of the initial launch state of the enclave, forming an \emph{attestation
report}; the customer can retrieve this report to verify the correct
software is running in an enclave.
%
To allow an attestation to bind some runtime value (typically a public
key), the enclave can include a small amount of \emph{user-data} in the
report; the user-data is opaque to the secure hardware, but otherwise covered
by the report's signature.
%
\emph{Sealing} enables an enclave to securely persist confidential data so that
only the enclave can decrypt it later, using a hardware-derived key tied to the
enclave's identity or measurement.


\parhead{Proxy Re-encryption}
%
% https://www.cs.jhu.edu/~susan/600.641/scribes/lecture17.pdf
%
Proxy re-encryption is a cryptographic scheme that allows an untrusted
proxy to convert a ciphertext encrypted under Alice's public key into a
ciphertext that Bob can decrypt with his secret key, without learning the
underlying plaintext.
%
At a high-level, Alice and Bob construct a public \emph{re-encryption key}
$RK_{\text{Alice}\rightarrow\text{Bob}}$ that
the proxy uses to re-encrypt the ciphertext from Alice (the \emph{delegator}) to Bob
(the \emph{delegatee}).
%
In 1998, Blaze, Bleumer, \& Strauss~\cite{98-eurocrypt-proxy_cryptography}
designed the first proxy re-encryption construction based on the ElGamal
encryption system~\cite{85-toit-elgamal}.
%
Shortly thereafter, Dodis and Ivan~\cite{03-ndss-proxy_cryptography_revisited}
developed a unidirectional variant, and later Ateniese et
al.~\cite{05-ndss-improved_proxy_reencryption} applied bilinear
maps~\cite{01-crypto-ibe_weil_pairing} (and specifically BLS
signatures~\cite{03-eurocrypt-aggregate_signatures_bilinear_maps}) to develop
schemes that did not require interaction between the delegator and delegatee.
%
Since then, numerous works have explored features like multiple
re-encryptions~\cite{17-tops-fast_proxy_re_encryption} and
revocation~\cite{12-crypto-dynamic_credentials_and_delegation_for_abe}, and
security properties like chosen-ciphertext
resistance~\cite{07-ccs-cca_proxy_re_encryption} and
unlinkability of ciphertexts~\cite{19-acisp-pcs_proxy_reencryption}.
