\section{Future Work}
\label{sec:future}

\textbf{adwait "discussion"}
% Discussion (discuss some of the important simplifying assumptions, and
% suggest possibilities for future work)

Notably, to get any benefit from SAMBA, users will have to use one of the three open source FaaS orchestrators (whichever we decide on), and run only Oak Restricted Kernel VMs.
It's doubtful that this will become the standard for FaaS workloads, since most in-use FaaS systems, such as AWS Lambda, use proprietary software.
SAMBA's practical in-market use cases are therefore likely next to none, but it serves as a good proof of concept for how to add these security features to FaaS systems.

As noted in the threat model, side channel attacks are out of scope for this project.
This assumption simplifies things greatly, since side-channel attacks are definitely feasible in the modern cloud. \cite{zhao_everywhere_2024} \cite{zhao_last-level_2024}
I acknowledge that it's a bit fantastical to assume that the cloud provider could be malicious, but woudlnt' think to perform a side-channel attack on our otherwise largely protected system.
Defenses against side-channel attacks in the cloud are another active research area.
It is unfortunate that strengthening overall security and hardeing against side-channel attacks seem to both leave the other vulnerable.
Uniting solutions for these two issues without significant performance impact is an interesting research area worthy of pursuit.
