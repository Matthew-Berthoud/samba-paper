Function as a Service (FaaS) is a popular cloud computing paradigm that
enables customers to execute individual functions in response to events, while
the cloud provider manages scaling and the underlying infrastructure.
%
Leveraging FaaS' simplicity and loose coupling, customers can create complex
computational pipelines by chaining the execution of multiple functions.
%
However, function chains often process sensitive user input, as for financial
transactions or machine learning inference, thus exposing private data to the
cloud provider and cloud attackers.
%
While secure hardware enclaves, such as Intel SGX, can protect function chains,
prior enclaved FaaS systems rely on a trusted key-provisioning enclave
to distribute encryption keys to the function instances.
%
Unfortunately,
such key provisioning enclaves constitute a key escrow, and thus a single target for
compromising the entire function chain.


In this thesis, we present our initial work on \Name, a confidential FaaS
system that removes the need for a trusted key provisioning service.
%
\Name runs each function instance in a secure enclave that locally generates
its encryption keys.
%
To allow the untrusted cloud provider to autoscale function instances without
introducing trusted services, \Name leverages cryptographic schemes for proxy
re-encryption, thus allowing the provider to safely re-encrypt data for any
instance replica.
%
We present our initial work in designing, implementing, and evaluating \Name's
proxy re-encryption protocol, and discuss future work in integrating \Name with
the popular Knative serverless platform.
