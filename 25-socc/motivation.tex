\section{Goals \& Assumptions}
\label{sec:motivation}

\parhead{Threat Model}
%
Our \SystemName system has three parties: the \emph{cloud provider}, the
\emph{function providers}, and \emph{external attackers}.
%
A function chain may contain functions from multiple providers.
%
For any given function provider, the potential adversaries
are the other function providers in the chain, the cloud provider, and
external attackers.
%
The goal of an adversary is to learn the inputs or outputs of a (peer)
function, modify these inputs or outputs, or modify the sequence of functions
in the chain (e.g., to insert a function that sends data to an
adversary-controlled log).
%
A function provider can submit any function to the chain, including a
malicious function that tries to leak data or subvert the cloud
provider.
%
We assume that functions may contain bugs that unintentionally leak
data or expose the function to exploitation.
%
We assume an adversary cannot breach the security of confidential VMs; we trust
the system software in the VM, and consider side-channel attacks out-of-scope.
%
%Since the cloud provider can trivially deny service, we do not guarantee
%availability.

\parhead{Goals}
%
\SystemName has a goal of enabling encryption between functions even in the
case of auto-scaling.
%
One goal of \SystemName-Lite is to help achieve \SystemName by implementing a
library which exposes methods that function replicas/instances can call to
encrypt data between them one another, with a limited trusted computing base
(TCB), and limited overhead.
%
Additionally, we strive to make \SystemName easy to use, so another goal of
\SystemName-Lite is to provide understandable examples and documentation for
the \SystemName library.

