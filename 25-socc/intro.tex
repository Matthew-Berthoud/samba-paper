\section{Introduction}
\label{sec:intro}
% Significance of the research and prior work.

% BIG PICTURE
%%%%%%%%%%%%%%%%%%%%%%%%%%%%%%%%%%%%%%%%%%%%%%%%%%%%%%%%%%%
%---
% - Define FaaS
% - Define Function Chain
% - Define characteristics (chain length, expected latency
%---
Function-as-a-Service (FaaS) platforms like AWS Lambda are popular choices for
building event-driven, cloud-based solutions because of their autoscaling
capabilities and pay-per-use billing.
%
%Functions are ephemeral, stateless units of code that are triggered by specific
%events, with the platform automatically scaling their execution up or down
%based on demand. 
%
The stateless nature and loose coupling of functions allow organizations to
create complex workflows simply by linking multiple functions in a sequence, or
\emph{function chain}.
%
A recent study found that 46\% of real-world FaaS applications use function
chains, with some chains containing up to 10
functions~\cite{20-atc-serverless_in_the_wild}.
%
%Although functions can directly invoke each other to form chains, they more
%commonly rely on intermediary services—such as event queues, pub/sub channels,
%or object stores—to trigger and pass control to the next function.
%%
%This indirect chaining supports loose coupling and circumvents input size
%limitations.
%
Despite the name, function chains may be input-dependent, allowing for branching
and looping, and are thus more accurately graphs.



% PROBLEM STATEMENT
%
% What is that you address and that keeps the big picture form being pretty?
% How would the world be better if the problem were solved?
%%%%%%%%%%%%%%%%%%%%%%%%%%%%%%%%%%%%%%%%%%%%%%%%%%%%%%%%%%%
%---
% - Describe use-cases that involve multiple parties and that are security sensitive
% - Give examples of concrete attacks
% - Give examples of cloud provider leaks, insider threat, law enforcement
% - Give examples of issues with constructing chains accross distrustful parties
% - End with problem statement/thesis in bold
%---
There are many use cases---particularly in highly-regulated sectors like
government, finance, and healthcare---, where function chains must process
privacy-sensitive data, and where the chain includes functions from multiple
organizations.
%
For instance, a chain to process a loan application could include: \wcircled{1}
the bank's initial function to process the application, \wcircled{2} a
government service's function  to verify the applicant's identity, and
\wcircled{3} a credit bureau's function to retrieve the applicant's credit
score.
%
In such a setting, each organization must trust another's function with their
data.
%
% [37, 44, 50]
Even with mutual trust among organizations, adversaries can exploit
vulnerabilities in functions, manipulating data flows to steal sensitive data
and conduct stealthy operations~\cite{21-sec-sandtrap}.
%
%For example, SandTrap~\cite{21-sec-sandtrap}
%demonstrates how an adversary can bypass IFTT rule sandboxes on AWS Lambda
%bypassed to exfiltrate IoT event data.
%
Moreover, despite seemingly trustworthy  cloud providers, customers must still
be wary of cloud infrastructure bugs~\cite{14-socc-cloud_bugs}, insider
threats~\cite{09-hotcloud-trusted_cloud_computing}, and data disclosures to law
enforcement~\cite{ amazon-ring-videos-police, police-microsoft-personal-data,
police-phone-data-icloud}.


% CHALLENGES IN SOLVING IT
%
% Why have people not solved this before?  What primitivies or mechanisms would
% one need to assemble a solution.  Can prior attempts at solutions not be
% deployed incrementally?  Are existing tools mistmathced to the "real"
% problem?
%%%%%%%%%%%%%%%%%%%%%%%%%%%%%%%%%%%%%%%%%%%%%%%%%%%%%%%%%%%
%---
% Describe three major prior solutions:
%   - commerical tee solutions (dataproc)
%   - academic tee solutions (clemmys)
%   - non-tee solutions (kalim)
%---
Prior work in securing function chains focuses either on guaranteeing the
chain's data confidentiality, or the integrity of function order.
%
For data confidentiality, previous research systems~\cite{16-osdi-ryoan,
    19-ccsw-s_faas,
    19-systor-clemmys,
    19-systor-trust_more_serverless,
    21-isca-confidential_serverless,
    23-sec-confidential_serverless,
    23-socc-cryonics}
leverage hardware trusted execution environments (TEEs, particularly Intel SGX
enclaves~\cite{13-hasp-innovative_instructions, 16-techreport-intel_sgx_explained})
to enable confidential function workflows.
%
Unfortunately, to support autoscaling, these systems rely on a trusted
key distribution enclave to provision each function replica with the same
keying material, creating a key escrow and single point of failure.
%
To ensure function order integrity, research systems like
Kalium~\cite{23-sec-kalium} and Valve~\cite{20-www-valve}
model the function chain as a call graph and enforce control-flow integrity to
guarantee that runtime execution follows the expected sequence.
%
Since a function has only a local view of the application, these systems also
must rely on a trusted, global controller to track the application-wide control
flow.


In this thesis, we demonstrate initial work in affirming our research
statement:
%
\begin{tcolorbox}[boxsep=1pt]
\textbf{
    A FaaS platform can guarantee the confidentiality and integrity of function
    chains without resorting to centralized trusted services.
}
\end{tcolorbox}

% YOUR INSIGHT
%
% What makes what you have written interesting?  What nugget of wisdom does the
% rest of the paper investigate?  One of your sentences should start with "Our
% insight is."
%%%%%%%%%%%%%%%%%%%%%%%%%%%%%%%%%%%%%%%%%%%%%%%%%%%%%%%%%%%

We present \SystemName, the first FaaS platform to guarantee data
confidentiality and chain integrity without relying on a trusted control plane.
%
\SystemName combines non-interactive cryptographic protocols
with TEEs to secure function chains without centralized trust.
%
In \SystemName, each function replica runs in a TEE and independently generates
its own keys.
%
Using proxy re-encryption, \SystemName enables any replica to decrypt
messages intended for its target function, eliminating the need for a trusted
key escrow.
%
For control flow integrity, Samba's runtime cryptographically signs each
function's provenance, removing the need for a global controller.
%
%To minimize bandwidth and validation overhead, Samba uses aggregate
%signatures~\cite{03-eurocrypt-aggregate_signatures_bilinear_maps} to compress
%the signatures of the entire chain into a single, compact signature.


% TODO: what minor challenges did you have to solve


%SystemName-Lite\footnote{\url{https://github.com/etclab/samba/tree/main/cmd}}
%Go\footnote{\url{https://github.com/etclab/pre}}, 
% \SystemName \footnote{\url{https://github.com/etclab/samba}} library 

\parhead{Contributions.}
%
We make the following contributions:
%
\begin{widelist}
\item We introduce \SystemName, a FaaS platform for ensuring the
    confidentiality and integrity of complex FaaS pipelines.
    %
    \SystemName composes trusted execution environments with a novel
    cryptographic protocol for proxy re-encryption to protect inter-function
    I/O while preserving horizontal scalability.

\item We implement an initial prototype of \SystemName that demonstrates the
    feasibility of proxy re-encryption in a FaaS setting.

\item We conduct a preliminary evaluation of \SystemName using several
    micro-benchmarks that measure the performance of proxy re-encryption 
    operations relative to a na\"{i}ve RSA baseline.
\end{widelist}


\parhead{Outline.}
%
This paper is organized as follows.
%
In \S\ref{sec:background} we provide an overview of FaaS, TEEs, and proxy
re-encryption.
%
In \S\ref{sec:goals} we specify our threat model, goals, and assumptions.
%
We present the design of \SystemName in \S\ref{sec:design}, our preliminary
implementation in \S\ref{sec:implementation}, and an evaluation of the overhead
of proxy re-encryption in \S\ref{sec:evaluation}.
%
We describe future work in extending our prototype in \S\ref{sec:future}, and
conclude in \S\ref{sec:conclusion}.
